\section{Serwer OPC}
\subsection{Wprowadzenie}
Systemy typu PLC – SCADA są powszechnie wdrażane poczynając od przemysłu chemicznego, a kończąc na automatyce budynków. Stanowią pewną normę w nowoczesnych zakładach oraz fabrykach. W związku z bogatą ofertą producentów aparatury automatyzacji powstał problem komunikacji pomiędzy różnymi komponentami. Firmy promowały swoje rodzime protokoły przemysłowe, co wymuszało na końcowych użytkownikach stosowanie sprzętu pochodzącego od tego samego producenta w obrębie całego obiektu. Przełomem okazało się wprowadzenie otwartego rozwiązania – standardu OPC. OPC jest standardem umożliwiającym komunikację pomiędzy sterownikami (najczęściej PLC) a oprogramowaniem SCADA. Bazuje na modelu klient-serwer, przy czym strona serwera zaimplementowana jest w oprogramowaniu dostarczanym przez producentów sterowników, natomiast klientem - aplikacja wykorzystująca udostępniane dane. 
\subsection{MatrikonOPC}
Dla celów zintegrowania omawianego systemu zastosowano testowy serwer udostępniony przez firmę Matrikon służący do celów niekomercyjnych. Dostawca oferuje serwer OPC (OPC Simulation Server), jak również oprogramowanie do zarządzania odczytywanymi danymi (OPC Explorer). 
%TODO konfiguracja serwera
\subsection{Logowanie OPC - MySQL}
Kolejny element stworzonego wielopoziomowego systemu sterowania stanowi aplikacja logująca aktualne dane pochodzące z serwera OPC do procesowej bazy danych. Aplikacja napisana została w języku C\#. Do komunikacji z serwerem użyto open sourcową bibliotekę TitaniumAS.Opc (https://github.com/titanium-as/TitaniumAS.Opc.Client). Serwer lokalizowany jest jedynie po nazwie, co, zgodnie z ideą OPC, znacząco przyspiesza proces integracji. Aplikacja odczytuje bieżące pomiary z serwera OPC z zadaną przez użytkownika częstotliwością, a następnie loguje je w bazie danych MySQL. Podczas tej operacji aktualne wartości porównywane są z progami alarmowymi zadanymi przez użytkownika. Ewentualne alarmy zapisywane są do bazy danych. Podstawową częstotliwością odpytywania jest 1 sekunda, tak jak to ma miejsce w większości systemów typu SCADA. Poszczególne pomiary identyfikowane są po tagach jakie zostały im nadane w momencie inicjalizacji w serwerze OPC. Zapis do bazy danych odbywa się według konwencji narzuconej w momencie jej zaprojektowania. Aplikacja zapewnia mechanizmy przechwytywania zgłaszanych błędów, w szczególności błędów połączenia z serwerem oraz MySQL. Użytkownik informowany jest o napotkanym błędzie. Program dokonuje niezbędnej konwersji sposobu zapisu liczb zmiennoprzecinkowych, zamienia separator ‘,’ na ‘.’ bez czego zapis do bazy danych nie byłby możliwy.
 
Warto nadmienić, że użytkownik operujący na systemie Windows nie musi instalować, żadnego dodatkowego oprogramowania w celu uruchomienia omawianej aplikacji logującej.
