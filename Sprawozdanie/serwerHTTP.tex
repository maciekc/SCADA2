\chapter{Serwer HTTP}
\section{Opis działania}
W celu umożliwienia komunikacji aplikacji wizualizującej z bazą danych napisano w języku \textit{JAVA} serwer HTTP. Dane pomiędzy serwerem i aplikacją wymieniane są za pomocą protokołu HTTP, wykorzystując w tym celu zapytania typu \textit{GET} i \textit{POST}. Wszystkie żądania są utożsamione z odpowiednim adresem url i odpowiednio przetwarzane po stronie serwerowej. Aby zapewnić jak największą responsywność aplikacji oraz synchroniczne odświeżanie danych każdy request po stronie serwera przetwarzany jest w osobnym wątku. \\
Do napisania serwera wykorzystano następujące biblioteki \textit{JAV-wy}:
\begin{enumerate}
	\item \textbf{RXJava} - biblioteka zapewniająca mechanizmy asynchronicznego przetwarzania danych \cite{rxjava},
	\item \textbf{JDBI} - biblioteka zapewniająca połączenie z bazą danych \cite{jdbi},
	\item \textbf{AKKA-HTTP} - framework do implementacji serwera HTTP \cite{akka doc},
	 \item \textbf{Javax mail} - biblioteka do wysyłania emaili.
\end{enumerate}

\section{Uruchomienie serwera}
W momencie pisania sprawozdania w repozytorium \textit{} nie udostępniono jeszcze skompilowanej wersji flików \'zródłowych. Dlatego też w celu uruchomienia aplikacji na komputerze wymagane jest: 
\begin{enumerate}
	\item zainstalowanie Jav-y w wersji 8
	\item zainstalowanie dowolnego środowiska programistycznego \textit{Intellij, Eclipse itp.}
	\item skompilowanie i uruchomienie całego projektu dostępnego pod adresem \\ \textit{https://github.com/maciekc/SCADA2}
\end{enumerate}