\chapter{Podsumowanie i wnioski}

Głównym celem projektu było stworzenie aplikacji umożliwiającej zdalny podgląd i zarządzanie wybranym procesem przemysłowym. Na wstępnym etapie zdecydowano, że obiektem, którego praca będzie nadzorowana zostanie model instalacji trzech połączonych ze sobą zbiorników. Przyjęto, że celem pracy układu sterowania w rozważanym systemie będzie nadzór nad mieszaniem cieczy w jednym ze zbiorników i kontrola poziomów w każdym z osobna. Na tej bazie zaproponowano układ sterujący opisany w rozdziale \ref{model_systemu}.\\ 
%
Uwzględniając specyfikę wybranego systemu sterowania zdecydowano się na napisanie aplikacji umożliwiającej podgląd wybranych zmiennych w czasie rzeczywistym jak i również zmianę wybranych parametrów. Aby umożliwić korzystanie z programu przez wielu użytkowników, oraz zapewnić łatwość obsługi, aplikacja napisana została w formie webowej, uruchamianej na dowolnej  przeglądarce internetowej. Dodatkowo chciano uwzględnić w architekturze całego systemu elementy charakterystyczne dla tego typu aplikacji tzn. komunikację sterownika z obiektem czy logowanie danych procesowych w bazie danych. \\
% 
W trakcie pracy wyniknęło kila problemów związanych głównie z integracją poszczególnych węzłów systemu. Największy z nich dotyczył zapisu poszczególnych danych w serwerze OPC, tak aby mogły być one dostępne w programie \textit{Simulink} oraz logowania parametrów systemu w bazie danych. Początkowo zakładano, że serwer \textit{HTTP} będzie odpowiedzialny za komunikacje z serwerem OPC, jednak w wyniku problemów z dostępnością biblioteki umożliwiającej poprawne nawiązanie komunikacji z OPC zdecydowano się napisać tę część systemu w języku \textit{C\#}. \\ 
Podsumowując, wszystkie założenia dotyczące zostały finalnie zrealizowane. Realizacja niniejszego problemu  była okazją do wykorzystania wiedzy zdobytej na przestrzeni całych studiów. Wykorzystane zostało nie tylko doświadczenie w projektowaniu układów regulacji w środowisku \textit{Matlab/Simulik} ale również umiejętności programowania w językach typu \textit{JAVA, C\#, Typescript}. Cennym doświadczeniem okazało się również zaprojektowanie struktury bazy danych, tak aby odzwierciedlała najważniejsze funkcjonalności systemu.